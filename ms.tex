
%%\documentclass[12pt,preprint]{aastex}

%% manuscript produces a one-column, double-spaced document:

%% \documentclass[10pt,manuscript]{aastex}

%% preprint2 produces a double-column, single-spaced document:
\documentclass[preprint2,apjl,numberedappendix,twocolappendix,appendixfloats]{emulateapj}
%% \documentclass[preprint2,iop]{aastex}

%% \documentclass[preprint2,longabstract]{aastex}

%% \usepackage{ccaption}
%% \captionstyle{\raggedright}
\usepackage[caption=false]{subfig}
\usepackage{amsmath}
\usepackage{footnote}
\bibpunct{(}{)}{;}{a}{}{,} 
\captionsetup{belowskip=12pt,aboveskip=4pt}
\setlength{\textfloatsep}{10pt plus 1.0pt minus 2.0pt}
\newcommand{\dif}{\mathrm{d}}
%% \renewcommand*{\thefootnote}{\fnsymbol{footnote}}

\def\nar{{New~A~Rev.}}          % New Astronomy Review
\def\pasa{{PASA}}               % Publications of the Astron. Soc. of Australia

%% \bibliographystyle{mn2e}
%% \bibliographystyle{apj}

\shorttitle{Wide--field Effects in EoR Power Spectra}
\shortauthors{Thyagarajan et~al.}

\def\ASU{\altaffilmark{1}}
\def\ASUtxt{\altaffiltext{1}{Arizona State University, School of Earth and Space Exploration, Tempe, AZ 85287, USA}}

\def\myemail{\altaffilmark{*}}
\def\myemailtxt{\altaffiltext{*}{e-mail: t\_nithyanandan@asu.edu}}

\def\UW{\altaffilmark{2}}
\def\UWtxt{\altaffiltext{2}{University of Washington, Department of Physics, Seattle, WA 98195, USA}}

\def\SKASA{\altaffilmark{3}}
\def\SKASAtxt{\altaffiltext{3}{Square Kilometre Array South Africa (SKA SA), Park Road, Pinelands 7405, South Africa}}

\def\RU{\altaffilmark{4}}
\def\RUtxt{\altaffiltext{4}{Department of Physics and Electronics, Rhodes University, Grahamstown 6140, South Africa}}

\def\CfA{\altaffilmark{5}}
\def\CfAtxt{\altaffiltext{5}{Harvard-Smithsonian Center for Astrophysics, Cambridge, MA 02138, USA}}

\def\ANU{\altaffilmark{6}}
\def\ANUtxt{\altaffiltext{6}{Australian National University, Research School of Astronomy and Astrophysics, Canberra, ACT 2611, Australia}}

\def\CAASTRO{\altaffilmark{7}}
\def\CAASTROtxt{\altaffiltext{7}{ARC Centre of Excellence for All-sky Astrophysics (CAASTRO)}}

\def\Haystack{\altaffilmark{8}}
\def\Haystacktxt{\altaffiltext{8}{MIT Haystack Observatory, Westford, MA 01886, USA}}

\def\MIT{\altaffilmark{9}}
\def\MITtxt{\altaffiltext{9}{MIT Kavli Institute for Astrophysics and Space Research, Cambridge, MA 02139, USA}}

\def\Curtin{\altaffilmark{10}}
\def\Curtintxt{\altaffiltext{10}{International Centre for Radio Astronomy Research, Curtin University, Perth, WA 6845, Australia}}

\def\Victoria{\altaffilmark{11}}
\def\Victoriatxt{\altaffiltext{11}{Victoria University of Wellington, School of Chemical \& Physical Sciences, Wellington 6140, New Zealand}}

\def\UWisc{\altaffilmark{12}}
\def\UWisctxt{\altaffiltext{12}{University of Wisconsin--Milwaukee, Department of Physics, Milwaukee, WI 53201, USA}}

\def\UMichigan{\altaffilmark{13}}
\def\UMichigantxt{\altaffiltext{13}{University of Michigan, Department of Atmospheric, Oceanic and Space Sciences, Ann Arbor, MI 48109, USA}}

\def\UMelbourne{\altaffilmark{14}}
\def\UMelbournetxt{\altaffiltext{14}{The University of Melbourne, School of Physics, Parkville, VIC 3010, Australia}}

\def\USydney{\altaffilmark{15}}
\def\USydneytxt{\altaffiltext{15}{The University of Sydney, Sydney Institute for Astronomy, School of Physics, NSW 2006, Australia}}

\def\CASS{\altaffilmark{16}}
\def\CASStxt{\altaffiltext{16}{CSIRO Astronomy and Space Science (CASS), PO Box 76, Epping, NSW 1710, Australia}}

\def\Tata{\altaffilmark{17}}
\def\Tatatxt{\altaffiltext{17}{National Centre for Radio Astrophysics, Tata Institute for Fundamental Research, Pune 411007, India}}

\def\RRI{\altaffilmark{18}}
\def\RRItxt{\altaffiltext{18}{Raman Research Institute, Bangalore 560080, India}}

\def\NRAO{\altaffilmark{19}}
\def\NRAOtxt{\altaffiltext{19}{National Radio Astronomy Observatory, Charlottesville and Greenbank, USA}}

\def\UWA{\altaffilmark{20}}
\def\UWAtxt{\altaffiltext{20}{International Centre for Radio Astronomy Research, University of Western Australia, Crawley, WA 6009, Australia}}

%% \definenote[thanks][conversion=set 2]

\begin{document}

\title{Wide--Field Effects in Redshifted 21~cm Power Spectra}

%% Use \author, \affil, and the \and command to format
%% author and affiliation information.
%% Note that \email has replaced the old \authoremail command
%% from AASTeX v4.0. You can use \email to mark an email address
%% anywhere in the paper, not just in the front matter.
%% As in the title, use \\ to force line breaks.

%% Author list
\author{
%% Lead Authors
Nithyanandan~Thyagarajan\ASU\myemail,
Daniel~C.~Jacobs\ASU,
Judd~D.~Bowman\ASU,
N.~Barry\UW,
A.~P.~Beardsley\UW,
G.~Bernardi\SKASA$^,$\RU$^,$\CfA,
F.~Briggs\ANU$^,$\CAASTRO,
R.~J.~Cappallo\Haystack, 
P.~Carroll\UW,
B.~E.~Corey\Haystack, 
% A.~A.~Deshpande\RRI, 
A.~de~Oliveira-Costa\MIT,
Joshua~S.~Dillon\MIT,
D.~Emrich\Curtin,
% B.~M.~Gaensler\USydney$^,$\CAASTRO, 
A.~Ewall-Wice\MIT,
L.~Feng\MIT,
R.~Goeke\MIT,
L.~J.~Greenhill\CfA,
B.~J.~Hazelton\UW, 
J.~N.~Hewitt\MIT,
N.~Hurley-Walker\Curtin,
M.~Johnston-Hollitt\Victoria,
D.~L.~Kaplan\UWisc, 
J.~C.~Kasper\UMichigan$^,$\CfA, 
Han-Seek Kim\UMelbourne$^,$\CAASTRO,
P.~Kittiwisit\ASU,
E.~Kratzenberg\Haystack, 
E.~Lenc\USydney$^,$\CAASTRO,
J.~Line\UMelbourne$^,$\CAASTRO,
A.~Loeb\CfA,
C.~J.~Lonsdale\Haystack, 
M.~J.~Lynch\Curtin, 
B.~McKinley\UMelbourne$^,$\CAASTRO,
S.~R.~McWhirter\Haystack,
D.~A.~Mitchell\CASS$^,$\CAASTRO, 
M.~F.~Morales\UW, 
E.~Morgan\MIT, 
A.~R.~Neben\MIT,
D.~Oberoi\Tata, 
A.~R.~Offringa\ANU$^,$\CAASTRO, 
S.~M.~Ord\Curtin$^,$\CAASTRO,
Sourabh Paul\RRI,
B.~Pindor\UMelbourne$^,$\CAASTRO,
J.~C.~Pober\UW,
T.~Prabu\RRI, 
P.~Procopio\UMelbourne$^,$\CAASTRO,
J.~Riding\UMelbourne$^,$\CAASTRO,
A.~E.~E.~Rogers\Haystack, 
A.~Roshi\NRAO, 
N.~Udaya~Shankar\RRI, 
Shiv~K.~Sethi\RRI,
K.~S.~Srivani\RRI, 
R.~Subrahmanyan\RRI$^,$\CAASTRO, 
I.~S.~Sullivan\UW,
M.~Tegmark\MIT,
S.~J.~Tingay\Curtin$^,$\CAASTRO, 
C.~M.~Trott\Curtin$^,$\CAASTRO,
M.~Waterson\Curtin$^,$\ANU,
R.~B.~Wayth\Curtin$^,$\CAASTRO, 
R.~L.~Webster\UMelbourne$^,$\CAASTRO, 
A.~R.~Whitney\Haystack, 
A.~Williams\Curtin, 
C.~L.~Williams\MIT,
C.~Wu\UWA,
J.~S.~B.~Wyithe\UMelbourne$^,$\CAASTRO
}

%Institutional footnotes (typeset, then rearrange here to be in order)
\ASUtxt
\UWtxt
\SKASAtxt
\RUtxt
\CfAtxt
\ANUtxt
\CAASTROtxt
\Haystacktxt
\MITtxt
\Curtintxt
\Victoriatxt
\UWisctxt
\UMichigantxt
\UMelbournetxt
\USydneytxt
\CASStxt
\Tatatxt
\RRItxt
\NRAOtxt
\UWAtxt
\myemailtxt

%% \email{t\_nithyanandan@rri.res.in}

%% \clearpage

\begin{abstract}

Foreground emission is currently the primary limitation to detection of redshifted H{\sc i} emission from the epoch of reionization. Modern radio telescopes that target this cosmological signal are typically wide--field instruments. Through modeling of delay spectra measured between antenna pairs, it has recently emerged that wide--field measurements imprint a characteristic {\it pitchfork}--shaped signature in this Fourier domain. It is characterized by enhanced power from foreground emission mapped to regions near the horizon and plays a significant role in determining the contamination of the cosmological H{\sc i} signal. With MWA data sensitivity improved by coherently averaging snapshots aligned in local sidereal time across different observing nights, we confirm the prediction from modeling at $>5\sigma$ level. 

\end{abstract}
 
\keywords{cosmology: observations --- dark ages, reionization, first stars --- large-scale structure of universe --- methods: statistical --- radio continuum: galaxies --- techniques: interferometric}

\section{Introduction}\label{sec:intro}



\section{Wide--Field Effects in Delay Spectrum}\label{sec:wide-field}

\citet{thy15} have described in detail the effects of wide--field measurements as seen in the delay spectra of interferometer {\it visibilities}. Here, we give a brief overview of the wide--field signature predicted therein. 

The delay spectrum for a baseline vector, $\boldsymbol{b}$, is given by \citep{par12a,par12b,thy13,thy15}: 
\begin{align}
  \tilde{V}_b(\tau) &\equiv \int V_b(f)\,W(f)\,e^{i2\pi f\tau}\,\dif f,
\end{align}
with interferometer visibilities, $V_b(f)$, given by \citep{van34,zer38,tho01}:
\begin{align} 
  V_b(f) &= \iint\limits_\textrm{sky} A(\hat{\boldsymbol{s}},f)\,I(\hat{\boldsymbol{s}},f)\,W_\textrm{i}(f)\,e^{-i2\pi f\frac{\boldsymbol{b}\cdot\hat{\boldsymbol{s}}}{c}}\,\dif\Omega \\
         &= \iint\limits_\textrm{sky} \frac{A(\hat{\boldsymbol{s}},f)\,I(\hat{\boldsymbol{s}},f)}{\sqrt{1-l^2-m^2}}\,W_\textrm{i}(f)\,e^{-i2\pi f\frac{\boldsymbol{b}\cdot\hat{\boldsymbol{s}}}{c}}\,\dif l\,\dif m, 
\end{align}
where, $I(\hat{\boldsymbol{s}},f)$ and $A(\hat{\boldsymbol{s}},f)$ are the sky brightness and antenna's directional power pattern, respectively, as a function of frequency ($f$) and direction on the sky denoted by the unit vector $\hat{\boldsymbol{s}}\equiv (l,m,n)$, $W_\textrm{i}(f)$ denotes instrumental bandpass weights, $W(f)$ is a spectral weighting function that controls the transfer function in the delay transform, $\dif\Omega=(1-l^2-m^2)^{-1/2}\,\dif l\,\dif m$ is the solid angle element to which $\hat{\boldsymbol{s}}$ is the unit normal vector, and $c$ is the speed of light. $\tau=\boldsymbol{b}\cdot\hat{\boldsymbol{s}}/c$ is the geometric delay between antenna pairs measured relative to the zenith and provides a mapping to position on the sky.

In wide--field measurements, the steep rise in subtended solid angle near the horizon for a fixed delay bin size significantly enhances the integrated emission near the horizon delay limits. This is found to be true for diffuse emission even on wide antenna spacings because their foreshortening towards the horizon makes them sensitive to large angular scales that match the inverse of their foreshortened lengths. 

Typically, an antenna is most sensitive towards its primary field of view relative to the rest of the sky, which in conjunction with the aforementioned wide--field effects results in a characteristic ``pitchfork'' signature in the delay spectrum. 

Although there is marginal evidence for presence of this feature in the {\it zenith} snapshot presented in \citet{thy15}, the high level of thermal noise prevented a robust confirmation. In this paper, we confirm the {\it pitchfork} signature using deeper data from the MWA.

\section{The Murchison Widefield Array Observations}\label{sec:MWA}

The MWA observations and data analysis used in this study are already described in \citet{thy15}. In order to reduce thermal fluctuations while maintaining coherence, it is essential to average independent data sets obtained over the same region of sky with identical beamformer settings. Hence, we select a subset of MWA snapshots each of duration 112~seconds obtained over different nights which are aligned to within 72~seconds of each other in {\it local sidereal time} (LST) around a mean LST of 0.04~hours with the MWA tile beam pointed at zenith. The database contains 14 snapshots satisfying these criteria. Two of these snapshots were found to contain amplitude and phase artifacts for a significant duration across different baselines. Hence, they have been excluded from our analysis. 

The delay spectra of the rest of the snapshots were verified to be coherent in their amplitudes and phases. These complex valued delay spectra from independent snapshots are averaged together to lower thermal fluctuations without losing coherence from foreground contributions. The results are discussed below.

\section{Results}\label{sec:results}

\section{Summary}\label{sec:summary}

\acknowledgments

This work was supported by the U. S. National Science Foundation (NSF) through award AST--1109257. DCJ is supported by an NSF Astronomy and Astrophysics Postdoctoral Fellowship under award AST--1401708. JCP is supported by an NSF Astronomy and Astrophysics Fellowship under award AST-1302774. This work makes use of the Murchison Radio-astronomy Observatory, operated by CSIRO. We acknowledge the Wajarri Yamatji people as the traditional owners of the Observatory site. Support for the MWA comes from the NSF (awards: AST-0457585, PHY-0835713, CAREER-0847753, and AST-0908884), the Australian Research Council (LIEF grants LE0775621 and LE0882938), the U.S. Air Force Office of Scientific Research (grant FA9550-0510247), and the Centre for All-sky Astrophysics (an Australian Research Council Centre of Excellence funded by grant CE110001020). Support is also provided by the Smithsonian Astrophysical Observatory, the MIT School of Science, the Raman Research Institute, the Australian National University, and the Victoria University of Wellington (via grant MED-E1799 from the New Zealand Ministry of Economic Development and an IBM Shared University Research Grant). The Australian Federal government provides additional support via the Commonwealth Scientific and Industrial Research Organisation (CSIRO), National Collaborative Research Infrastructure Strategy, Education Investment Fund, and the Australia India Strategic Research Fund, and Astronomy Australia Limited, under contract to Curtin University. We acknowledge the iVEC Petabyte Data Store, the Initiative in Innovative Computing and the CUDA Center for Excellence sponsored by NVIDIA at Harvard University, and the International Centre for Radio Astronomy Research (ICRAR), a Joint Venture of Curtin University and The University of Western Australia, funded by the Western Australian State government.  

% \par\bigskip
\bibliographystyle{apj}
\bibliography{eor}

\end{document}
